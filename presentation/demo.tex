\documentclass[10pt,table]{beamer}

\usetheme[progressbar=frametitle]{metropolis}
\usepackage{appendixnumberbeamer}

\usepackage{booktabs}
\usepackage[scale=2]{ccicons}
\usepackage{media9}
\usepackage{multimedia}
\usetikzlibrary{positioning,arrows.meta,chains,fit, trees}
\usepackage{pgfplots}
\usepackage{soul}
\usepgfplotslibrary{dateplot}
\usepackage{multicol}
\usepackage{xspace}
\usepackage{setspace}
\usepackage{changepage}
\usepackage[export]{adjustbox}
\usepackage{subcaption}
\usepackage{wasysym}
\usepackage[10pt]{moresize}
\usepackage{ragged2e}
\usepackage{bigdelim}
\usepackage{fancyvrb}

\usepackage{minted}

% Global options for minted
\setminted[r]{
  frame=single,
  fontsize=\scriptsize,
  breaklines,
  autogobble,
  label=R Code,
  style=tango, % Change the style as needed
}

\newcommand{\themename}{\textbf{\textsc{metropolis}}\xspace}
\setbeamertemplate{caption}{\raggedright\insertcaption\par}

\usepackage{tikz}
\usetikzlibrary{shapes,arrows,positioning,calc,arrows.meta}
\usetikzlibrary{decorations}
\usetikzlibrary{decorations.markings}
\usetikzlibrary{decorations.pathreplacing}

\usepackage{relsize}
\tikzset{fontscale/.style = {font=\relsize{#1}}
    }
\tikzset{>=latex}
\usetikzlibrary{shapes,arrows}

\usepackage{framed}

\newcommand{\fullframegraphic}[1]{
\begin{frame}
\includegraphics[height=\textheight,width=\textwidth,keepaspectratio]{#1}
\end{frame}
}


% conditional for xetex or luatex
\newif\ifxetexorluatex
\ifxetex
  \xetexorluatextrue
\else
  \ifluatex
    \xetexorluatextrue
  \else
    \xetexorluatexfalse
  \fi
\fi
%

\newcommand*\quotesize{50} % if quote size changes, need a way to make shifts relative
% Make commands for the quotes
\newcommand*{\openquote}
   {\tikz[remember picture,overlay,xshift=-3ex,yshift=-1.5ex]
   \node (OQ) {\quotefont\fontsize{\quotesize}{\quotesize}\selectfont``};\kern0pt}

\newcommand*{\closequote}[1]
  {\tikz[remember picture,overlay,xshift=2.5ex,yshift={#1}]
   \node (CQ) {\quotefont\fontsize{\quotesize}{\quotesize}\selectfont''};}

% select a colour for the shading
\colorlet{shadecolor}{white}

\newcommand*\shadedauthorformat{\emph} % define format for the author argument

% Now a command to allow left, right and centre alignment of the author
\newcommand*\authoralign[1]{%
  \if#1l
    \def\authorfill{}\def\quotefill{\hfill}
  \else
    \if#1r
      \def\authorfill{\hfill}\def\quotefill{}
    \else
      \if#1c
        \gdef\authorfill{\hfill}\def\quotefill{\hfill}
      \else\typeout{Invalid option}
      \fi
    \fi
  \fi}
% wrap everything in its own environment which takes one argument (author) and one optional argument
% specifying the alignment [l, r or c]
%
\newenvironment{shadequote}[2][l]%
{\authoralign{#1}
\ifblank{#2}
   {\def\shadequoteauthor{}\def\yshift{-2ex}\def\quotefill{\hfill}}
   {\def\shadequoteauthor{\par\authorfill\shadedauthorformat{#2}}\def\yshift{2ex}}
\begin{snugshade}\begin{quote}\openquote}
{\shadequoteauthor\quotefill\closequote{\yshift}\end{quote}\end{snugshade}}

\usepackage{soul}
\makeatletter
\let\HL\hl
\renewcommand\hl{%
  \let\set@color\beamerorig@set@color
  \let\reset@color\beamerorig@reset@color
  \HL}
\makeatother

\usepackage{scalerel}
\usepackage{xparse}

\NewDocumentCommand\like{}{
    \includegraphics[scale=0.0175]{like.png}
}
\NewDocumentCommand\love{}{
    \includegraphics[scale=0.0175]{love.png}
}
\NewDocumentCommand\care{}{
    \includegraphics[scale=0.0175]{care.png}
}
\NewDocumentCommand\wow{}{
    \includegraphics[scale=0.0175]{wow.png}
}
\NewDocumentCommand\haha{}{
    \includegraphics[scale=0.0175]{haha.png}
}
\NewDocumentCommand\angry{}{
    \includegraphics[scale=0.0175]{angry.png}
}
\NewDocumentCommand\sad{}{
    \includegraphics[scale=0.0175]{sad.png}
}

% \setlength{\leftmargini}{2pt}
\setbeamersize{text margin left=15pt,text margin right=15pt}
\setbeamercolor{background canvas}{bg=white}

\title{The Dual Roles of Cognitive and Political Sophistication in Belief in Political (Mis)Information}
\author{\vspace{-0.75em} Gabrielle Péloquin-Skulski, Chloe Wittenberg, Adam Berinsky, Gordon \\ Pennycook, and David Rand}
\date{\vspace{0.5em} MPSA | April 5, 2024}

\begin{document}

\maketitle


\setbeamertemplate{section in toc}[ball unnumbered]



\begin{frame}[plain, label = two_dimensions]
\begin{tikzpicture}[remember picture, overlay]

\node[at=(current page.center),xshift=0cm,yshift=0cm](m1) {\includegraphics[keepaspectratio, width=1\paperwidth,height=\paperheight]{figures_crt_pk/headlines.pdf}};
\end{tikzpicture}

% CW note: my intended structure talking points, given these headlines, was something like:
% - Misinformation is a source of ample concern in the United States, accelerated by the rise of the Internet and social media
% - Although misinformation is prevalent across a variety of settings, these concerns are particularly acute in the political domain, given the risk that misinformation could distort policy preferences, erode citizens' trust in government, and threaten democracy
% - Given these dynamics, important to understand what factors predispose certain individuals to believe misinformation - and under what conditions
\end{frame}


\begin{frame}[t, fragile]{Two Dimensions of Political Information Processing}

\setlength{\leftmargini}{18pt}
\setlength{\rightmargini}{20pt}
\vspace{0.75em} 

\begin{enumerate}
    \item \alert{\bf Cognitive sophistication:} individuals' general propensity to engage in \textbf{deliberative} versus \textbf{intuitive} reasoning. \vspace{0.25em} \\ \scriptsize{\textcolor{gray}{(Evans \& Stanovich 2013; Pennycook et al. 2015; Toplak et al. 2014)}} \normalsize  \pause  \bigskip
    \item \alert{\bf Political sophistication:} what people know and how they think about \textbf{politics}---including aspects of \textbf{knowledge}, \textbf{interest}, and \textbf{engagement}. \vspace{0.25em} \\ \scriptsize{\textcolor{gray}{(Delli Carpini \& Keeter 1996; Goren 2012; Luskin 1987, 1990)}} \normalsize \pause \bigskip
\end{enumerate}

\vspace{-1.5em}
\setbeamertemplate{blocks}[rounded][shadow=false]
\metroset{block=fill} 
\begin{center}
\begin{minipage}{8cm}
\begin{block}{Our Contribution}
    Rather than study these concepts in \textbf{isolation}, we integrate them into a shared \textbf{theoretical} and \textbf{empirical} framework.
\end{block}
\end{minipage}
\end{center}
    
\end{frame}

\section{Theoretical Foundations}

\begin{frame}[t, fragile, label=theory_cognitive]{Dimension \#1: Cognitive Sophistication}

\setlength{\leftmargini}{15pt}
\setlength{\rightmargini}{20pt}
\vspace{0.5em} \small

\begin{itemize}
    \item \alert{\bf Previous perspective:} Early work raises concerns that deliberation may \\ lead to \textbf{polarization} versus \textbf{consensus} in political beliefs by facilitating politically \textbf{motivated reasoning}. \vspace{0.25em} \\ \ssmall{\textcolor{gray}{(Drummond \& Fischhoff 2017; Kahan 2013, 2017; Kahan et al. 2012; Kuru et al. 2017)}} \small \bigskip \pause
    \item \alert{\bf Classical reasoning account:} More recent research, however, suggests \\ that deliberation can \textbf{enhance}, rather than \textbf{impede}, the formation  of \\ accurate beliefs. \vspace{0.25em} \\ \ssmall{\textcolor{gray}{(Bago et al. 2020, 2022; Pennycook \& Rand 2019, 2021; Ross et al. 2021)}} \small \bigskip \pause 
    \item \alert{\bf Link to misinformation:} In line with this view, analytic thinking tends to be \\ associated with reduced belief in \textbf{fake news} and \textbf{conspiracy theories} and \\ improved ability to discern whether information is \textbf{true} or \textbf{false}. \vspace{0.25em} \\ \ssmall{\textcolor{gray}{(Bronstein et al. 2019; Pennycook \& Rand 2019; Pennycook et al. 2020; Swami et al. 2014)}} \small \medskip 
\end{itemize}

    % If want to incorporate MS2R account, could say something like "Early work raised concerns that deliberation may increase individuals' susceptibility to misinformation, exacerbate polarization by promoting motivated reasoning. (Kahan 2013, 2017; Kahan et al. 2012; Knobloch-Westerwick et al. 2020)
    % Then, can say, "More recent research, however, finds that people fall for misinformation when they \textit{fail} to deliberate" (Pennycook \& Rand 2019, etc.)
    % In addition, reference work showing that analytic thinking improves discernment of true vs. false content, \textit{regardless} of political slant (suggesting that cognitive sophistication may reduce people's likelihood of believing false claims, increase likelihood of believing true claims, regardless of content type)
\end{frame}

\begin{frame}[t, fragile]{Dimension \#2: Political Sophistication}

    \setlength{\leftmargini}{15pt}
\setlength{\rightmargini}{20pt}
\vspace{0.5em} \small

\begin{itemize}
    \item \alert{\bf Overall patterns:} \textbf{On the whole}, political sophistication may improve \\ people's ability to judge whether information is true or false by furnishing domain-specific \textbf{knowledge} and \textbf{contextual understanding}. \vspace{0.25em} \\ \ssmall{\textcolor{gray}{(Berinsky 2023; Goren 2012; Luskin 1987; Vegetti \& Mancosu 2020)}} \small \bigskip \pause
    \item \alert{\bf Role of concordance:} However, higher levels of political \textbf{knowledge} and \textbf{awareness} tend to be associated with \textbf{increased} belief in \\ claims that align with one's \textbf{partisan identity}. \vspace{0.25em}\\ \ssmall{\textcolor{gray}{(Flynn et al. 2017; Jardina \& Traugott 2018; Kahne \& Bowyer 2017; Nyhan et al. 2013; Zaller 1992)}} \small \bigskip \pause
    \item \alert{\bf Link to misinformation:} Although political sophistication may \textbf{improve} truth \textbf{discernment} overall, this pattern may \textbf{differ} for \textbf{concordant} and \textbf{discordant} information. \vspace{0.75em} \\ \ssmall{\textcolor{gray}{(Berinsky 2012, 2023; Miller et al. 2016; Vitriol et al. 2023)}}  
\end{itemize}

    % First bullet: overall, pol sophistication expected to improve people's ability to discern when information is true or false (link to both empirical lit, as well as intuition from interventions)
    % However, across variety of domains, political knowledge is associated with biased processing of information based on one's partisan and ideological attachments
    % Effects of political sophistication likely context-dependent - lay out expectations
\end{frame}


\begin{frame}[t, fragile]{Integrating Cognitive and Political Sophistication}

    \setlength{\leftmargini}{15pt}
\setlength{\rightmargini}{20pt}
\vspace{1em} \small

\begin{itemize}
    \item To date, \alert{\bf cognitive} and \alert{\bf political} sophistication have been largely studied in \alert{\bf isolation}. \vspace{0.25em} \\ \ssmall{\textcolor{gray}{(e.g., Flynn et al. 2017; Pennycook \& Rand 2019, 2021; Nyhan et al. 2013; though see Vitriol et al. 2023)}} \small \bigskip \pause
    \item But these two traits tend to be closely \alert{\bf intertwined}---both \alert{\bf conceptually} and \alert{\bf empirically}. %Both traits are \textbf{correlated} with \textbf{education} level, and \textbf{cognitive} sophistication may help people analyze, interpret, and retain \textbf{political information}.
    \vspace{0.25em} \\ \ssmall{\textcolor{gray}{(Delli Carpini and Keeter 1996; Highton 2009; Stieger and Reips 2016)}} \small \bigskip \pause
    \item Jointly modeling these two factors may therefore help reconcile \alert{\bf conflicting findings} from prior research. \medskip \vspace{0.25em} \\
        \begin{itemize}
            \item[-] Example: to what extent is the positive relationship between \textbf{political knowledge} and truth \textbf{discernment} confounded by \textbf{analytic thinking}?
        \end{itemize}
    % \item \alert{\bf Implication:} Thus, the \textbf{conflicting findings} from prior research may stem from the fact that cognitive and political sophistication are \textbf{associated} with one another. 
\end{itemize}

\end{frame}


% \begin{frame}[t, fragile]{Cognitive and Political Foundations of Belief in (Mis)information}


% \setlength{\leftmargini}{5pt}
% \setlength{\rightmargini}{25pt}
% \vspace{0.5em} \small

% \begin{itemize}
% \item[]  \alert{\bf Cognitive Sophistication:}  \vspace{0.25em} \pause
% \begin{itemize}
%         \item According to theories of \textbf{classical reasoning}, people are more likely to endorse false claims when they do not deliberate and rely on their initial intuition.
%         \ssmall{\textcolor{gray}{(Bronstein et al. 2019; Pennycook and Rand 2019; Salvi et al. 2023)}} \small  \pause 
%         \item Greater cognitive sophistication leads to greater ability to discern whether political information is true or false, regardless of whether this information aligns with people's prior beliefs. \ssmall{\textcolor{gray}{(Pennycook and Rand 2019; Pennycook et al. 2020)}} \small 
%     \end{itemize} \pause
% \end{itemize}
% \begin{quote} 
%         \item \textbf{Hypothesis 1 ($\text{H}_1$):} Overall, higher levels of cognitive sophistication will be associated with enhanced discernment of true versus false content, as evidenced by reduced belief in false claims and increased belief in true claims. \pause \vspace{0.5em}
%         \item \textbf{Hypothesis 2 ($\text{H}_2$):} This relationship between cognitive sophistication and perceived accuracy will persist for both concordant and discordant messages.
% \end{quote}
% \end{frame}

% \begin{frame}[t, fragile]{Cognitive and Political Foundations of Belief in (Mis)information}


% \setlength{\leftmargini}{5pt}
% \setlength{\rightmargini}{25pt}
% \vspace{0.5em} \small

% \begin{itemize}
% \item[]  \alert{\bf Political Sophistication:} \pause
% \begin{itemize}
%         \item Political
% sophistication can enhance individuals’ ability to detect whether information is true or false, as those who score higher on this trait have greater political knowledge.       \ssmall{\textcolor{gray}{(Berinsky 2023; Vegetti and Mancosu 2020)}} \pause \small 
%         \item But, According to theories of politically motivated reasoning, more knowledgeable citizens have the ability and motivation to interpret new information in a biased way. 
% \ssmall{\textcolor{gray}{(Lodge and Taber 2013)}} \pause \small 
%     \end{itemize} 
% \end{itemize}
% \begin{quote} 
%         \item \textbf{Hypothesis 3 ($\text{H}_3$):} In general, higher levels of political sophistication will be associated with greater discernment of true versus false content, as indicated by reduced belief in false claims and increased belief in true claims.\pause \vspace{0.5em}
%         \item \textbf{Hypothesis 4 ($\text{H}_4$):} When it comes to \textit{mis}information, the relationship between political sophistication and perceived accuracy will be weaker for \textit{concordant} versus \textit{discordant} messages (and vice versa for \textit{true} claims). 
% \end{quote}

    
% \end{frame}





\section{Data and Methods}

\begin{frame}[t,fragile, label = data_overview]{Overview of Research Design}

\setlength{\leftmargini}{20pt}
\setlength{\rightmargini}{20pt}

\vspace{0.25em} \small

\begin{adjustwidth}{0.5em}{-0.25em}
We re-analyze data from \alert{\bf 12 studies} conducted online between 2019 and 2022 \\ (total \textit{N} = \alert{\bf 7,879 U.S. adults}; \alert{\bf 125,202 observations} spanning \alert{\bf 390 news headlines}). 
\end{adjustwidth} \pause \vspace{0.5em} 


\setlength{\columnsep}{-2cm}
\begin{itemize}
    \item {\bf Study design:} participants were asked to rate the accuracy of a series of \\ both \textit{true} and \textit{false} headlines about politics. \vspace{0.25em} \medskip \pause
    % \item \alert{\bf Headlines:} included nearly \textbf{400} social media posts varying in their \textbf{veracity} and \textbf{partisan slant}. \medskip \vspace{0.25em} \pause 
   \item {\bf Individual-level variables:} \vspace{0.5em}
        \begin{enumerate} 
\footnotesize
            \item \ul{Cognitive sophistication:} measured using the \textit{Cognitive Reflection Test} \textcolor{gray}{(CRT; Frederick 2005)}\vspace{0.5em}
            \item \ul{Political sophistication:} measured using \textit{political knowledge }scales \textcolor{gray}{(PK; Delli Carpini \& Keeter 1996)}
        \end{enumerate} \vspace{0.5em} \pause
    \item {\bf Content-level variables:} \vspace{0.5em}
    \begin{enumerate} \footnotesize
        \item \ul{Veracity:} whether the headline was \textit{true} or \textit{false/misleading} \vspace{0.5em}
        \item \ul{Concordance:} whether the headline \textit{aligns} or \textit{conflicts} with a participant's \\ \textit{partisan identity}
    \end{enumerate}
\end{itemize}



\vspace{0.85em}
\hspace{-0.75em}\hyperlink{studies}{\beamergotobutton{Summary of the Studies}}\hspace{0.5em}\hyperlink{example_head}{\beamergotobutton{Example Headlines}}\hspace{0.5em}\hyperlink{crt_wording}{\beamergotobutton{Survey Items}}

\end{frame}



% \begin{frame}[t, fragile, label = variables]{Variables of Interest}

% \setlength{\leftmargini}{5pt}
% \setlength{\leftmarginii}{8pt}
% \setlength{\leftmarginiii}{8pt}
% \setlength{\rightmargini}{10pt}
% \small
% %\vspace{0.5em}

% \begin{itemize} 
%     \item[] \alert{\bf Dependent Variable} 
%         \begin{itemize} \footnotesize
%             \item \textbf{Perceived Headline Accuracy}: ``To the best of your knowledge, is the claim in the above headline accurate?" OR ``What is the likelihood that the above headline is true?"
%         \end{itemize} 
%     \bigskip
%     \pause \small
%     \item[] \alert{\bf Individual-Level Variables:} 
%             \begin{itemize} \footnotesize
%             \item \textbf{Cognitive Sophistication}: Cognitive Reflection Test (CRT, Frederick 2005), series of logical questions, each of which has an intuitive but incorrect response.
%             \item \textbf{Political Sophistication}: Battery of questions measuring individuals’ political knowledge.
%             \item \textbf{Control Variables}: Age, gender, education, income. 
%         \end{itemize} 
%     \bigskip \pause  
%     \item[] \alert{\bf Content-Level Variables:} 
%         \begin{itemize} \footnotesize
%             \item \textbf{Veracity}: True or false (including misleading)
%             \item \textbf{Political Concordance}: Alignment between a respondent’s partisan identity (i.e., Democrat or Republican) and the slant of the presented headline (i.e., Democratic- or Republican-leaning). 
%         \end{itemize}     
% \end{itemize}

% \vspace{4em}

% \hspace{-1em}\hyperlink{dvs}{\beamergotobutton{Summary Statistics}}\hspace{0.5em}\hyperlink{flow}{\beamergotobutton{Sample Demographics}\hspace{0.5em}\hyperlink{flow}{\beamergotobutton{CRT Items}}\hspace{0.5em}\hyperlink{flow}{\beamergotobutton{PK Items}}}

% \end{frame}



\begin{frame}[t, fragile, label = empirics]{Empirical Strategy}

\setlength{\leftmargini}{5pt}
\setlength{\leftmarginii}{8pt}
\setlength{\leftmarginiii}{8pt}
\setlength{\rightmargini}{10pt}
\small
%\vspace{0.5em}


\setlength{\leftmargini}{15pt}
\setlength{\rightmargini}{30pt}
\vspace{0.5em} \small

\begin{itemize}
    \item We test our hypotheses using a Bayesian linear \alert{\textbf{mixed-effect}} model \\ predicting perceived \alert{\bf headline accuracy}. \medskip \pause 
    \item The model linearly interacts our \alert{\bf respondent-level} variables (e.g., CRT, PK) \\ with an indicator of headline \alert{\bf veracity} and an indicator of political \\ \alert{\bf concordance}. \medskip \pause
    \item We allow the \alert{\bf intercept} and \alert{\bf slopes} to vary by \alert{\bf study}, \alert{\bf headline} and \alert{\bf respondent}.
\end{itemize}
\medskip
\pause
\begin{minted}{r}
library(brms)

brm(accuracy ~ (pk + crt + age + gender + education + income + pid ) * real * concordant + 
               (1 + (pk + crt + ... ) * real * concordant | study) + 
               (1 + (pk + crt + ... ) * concordant | headline) + 
               (1 + real * concordant | respondent), 
    data = dat, ... )
\end{minted}




\end{frame}

%%%%%%%%%%%%%%%%%%%%%%%%%%%%%%%%%%%%%%%%%%%%%%%%%%%%%%%%%%%%%%%%%%%%%%
\section{Results}


\begin{frame}[fragile, label = acc_veracity]{Cognitive,  Political Sophistication Both Predict Discernment}

\begin{figure}
\begin{tikzpicture}[remember picture,overlay]
    \node<1>[at=(current page.center), yshift = -0.25cm]{
        \includegraphics[width=0.9\textwidth]{figures_mpsa/mpsa_truth_discernment_1.pdf}
        };
    \node<2->[at=(current page.center), yshift = -0.25cm]{
        \includegraphics[width=0.9\textwidth]{figures_mpsa/mpsa_truth_discernment_2.pdf}
        };   
\begin{pgfinterruptboundingbox}
            % the following coords. may need to be changed
            \fill <3> [fill=white, opacity=0.6, yshift = -3.5cm] (-7,4) rectangle (7,-5);
            \end{pgfinterruptboundingbox}
        \node <3> [draw, shape=rectangle, align=center, at=(current page.center), yshift = -0.2cm,fill=white,inner sep = 10pt, line width=0.5mm] {\Large \textbf{Takeaway:} Both \textbf{cognitive} and \textbf{political}  \\[1ex]\Large sophistication predict \textbf{greater} discernment  \\[1ex]\Large of \textbf{true} versus \textbf{false} content.};
    \end{tikzpicture} 
\end{figure} 

\vspace{19.5em}
\onslide<2->{\hspace{-0.75em}\hyperlink{perc_acc_veracity}{\beamergotobutton{Perceived Accuracy}}\hspace{0.5em}\hyperlink{dis_concord}{\beamergotobutton{Disaggregated by Concordance}}}


%\hspace{0.5em}\hyperlink{truth_discernment_conc}{\beamergotobutton{Truth Discernment: Concordance}}}
 
\end{frame}

%%%%%%%%%%%%%%%%%%%%%%%%%%%

\begin{frame}[t,fragile, label = acc_conc]{Political Sophistication Predicts Biased Information Processing}

\vspace{1.5em}
\begin{figure}
\begin{tikzpicture}[remember picture,overlay]
    \node<1>[at=(current page.center), yshift = -0.35cm]{
        \includegraphics[width=0.9\textwidth]{figures_mpsa/mpsa_bias_1.pdf}
        };
    \node<2->[at=(current page.center), yshift = -0.35cm]{
        \includegraphics[width=0.9\textwidth]{figures_mpsa/mpsa_bias_2.pdf}
        };
\begin{pgfinterruptboundingbox}
            % the following coords. may need to be changed
            \fill <3> [fill=white, opacity=0.6, yshift = -3.5cm] (-7,4.75) rectangle (10,-7);
            \end{pgfinterruptboundingbox}
        \node <3> [draw, shape=rectangle, align=center, at=(current page.center), yshift = -0.3cm,fill=white,inner sep = 10pt, line width=0.5mm] {\Large \textbf{Takeaway:} Only \textbf{political} sophistication is \\[1ex]\Large associated with \textbf{biased} information processing \\[1ex]\Large based on the political \textbf{concordance} of the content.};
    \end{tikzpicture} 
\end{figure} 

\vspace{17.7em}
\onslide<2->{\hspace{-0.75em}\hyperlink{decomp_veracity}{\beamergotobutton{Disaggregated by Veracity}}}

% \vspace{17.5em}
% \onslide<2->{\vspace{0.5em}
% \hspace{-0.75em}\hyperlink{trait<1>}{\beamergotobutton{Perceptions of the Label}}\hspace{0.5em}\hyperlink{learn_fx}{\beamergotobutton{Statistical Tests}}}

\end{frame}

%%%%%%%%%%%%%%%%%%%%%%%%%%%


\begin{frame}[t,fragile, label = acc_conc]{Decomposing Patterns by Veracity and Concordance}

\vspace{1.5em}
\begin{figure}
\begin{tikzpicture}[remember picture,overlay]
    \node<1>[at=(current page.center), yshift = -0.35cm]{
        \includegraphics[width=0.9\textwidth]{figures_mpsa/mpsa_full_plot.pdf}
        };
    \node<2->[at=(current page.center), yshift = -0.35cm]{
        \includegraphics[width=0.9\textwidth]{figures_mpsa/mpsa_full_plot_2.pdf}
        };
\begin{pgfinterruptboundingbox}
            % the following coords. may need to be changed
            \fill <3-4> [fill=white, opacity=0.6, yshift = -2.5cm] (-7,3.85) rectangle (7,-7);
            \end{pgfinterruptboundingbox}
        \node <3-4> [draw, shape=rectangle, align=center, at=((current page.center)+(0,-0.5cm)), yshift = -0.5cm,fill=white,inner sep = 10pt, line width=0.5mm] {\large \textbf{Takeaway \#1:} Higher \textbf{cognitive sophistication} is\\[1ex]\large associated with \textbf{enhanced truth discernment} for \\[1ex]\large both concordant \textbf{\textit{and}} discordant headlines.};

        % \node <3> [draw, shape=rectangle, align=center, at=((current page.center)+(0, -0.7cm)), yshift = -4cm,fill=white,inner sep = 10pt, line width=0.5mm] {\large \textbf{Takeaway 2:} The relationship between \textbf{political} \\[1ex]\large sophistication  and belief in \textbf{true} information is \\[1ex]\large \textbf{stronger} for \textbf{concordant} versus discordant headlines. \\[1ex]\large The opposite holds for misinformation.};
        \node <4> [draw, shape=rectangle, align=center, at=((current page.center)+(0, -0.5cm)), yshift = -4cm,fill=white,inner sep = 10pt, line width=0.5mm] {\large \textbf{Takeaway \#2:} The relationship between \textbf{political}\\[1ex]\large sophistication and belief in \textbf{misinformation} is \\[1ex]\large \textbf{weaker} for \textbf{concordant} vs. \textbf{discordant} headlines\\[1ex]\large (and vice versa for \textbf{true} content).};
    \end{tikzpicture} 
\end{figure} 


% \vspace{17.5em}
% \onslide<2->{\vspace{0.5em}
% \hspace{-0.75em}\hyperlink{trait<1>}{\beamergotobutton{Perceptions of the Label}}\hspace{0.5em}\hyperlink{learn_fx}{\beamergotobutton{Statistical Tests}}}

\end{frame}

\section{Takeaways and Discussion}

\begin{frame}[t,fragile, label = key_findings]{Summary of Key Findings}

\setlength{\leftmargini}{15pt}
\setlength{\rightmargini}{10pt}
\vspace{0.25em}

\textbf{Truth Discernment} \vspace{0.25em}
\begin{itemize}
\item Higher levels of \alert{\bf cognitive} and \alert{\bf political sophistication} are both associated with greater \alert{\bf truth discernment}.  
\end{itemize}
\bigskip \pause

\textbf{Bias} \vspace{0.25em}
\begin{itemize}
\item Overall, \alert{\bf concordant} information tends to be perceived to be more \alert{\bf accurate}. 
\bigskip \pause

\item But this bias appears to be \alert{\bf magnified} at higher levels of \alert{\bf political}, but \alert{\bf not} cognitive, sophistication.
\end{itemize}


% \item \alert{\bf Finding \#1:} Higher levels of \textbf{cognitive sophistication}, measured by analytic reasoning, are associated with \textbf{greater} belief in \textbf{true} and \textbf{lower} belief in \textbf{false} political content. 
% \onslide<3->{\item \alert{\bf Finding \#2:} This relationship holds for both concordant and discordant political information. 
% \bigskip \vspace{1em}} \pause
% \onslide<4>{\item \alert{\bf Finding \#3:} Higher levels of political sophistication, measured by political knowledge, are associated  with \textbf{greater} belief in \textbf{true} and \textbf{lower} belief in \textbf{false} political content. \bigskip \vspace{1em}} \pause

% \item \alert{\bf Finding \#4:} For true information, the relationship between political sophistication and perceived accuracy is stronger for concordant than for discordant content. The opposite holds for misinformation. 


\end{frame}

\begin{frame}[t,fragile, label = discussion]{Questions for Discussion}

% Some limitations are analytical, some more fundamental

\setlength{\leftmargini}{15pt}
\setlength{\rightmargini}{20pt}
\vspace{0.5em} 
\small

\textbf{Limitations \& Extensions} \vspace{0.25em}
\begin{itemize}
    \item \alert{\textbf{Correlational versus causal: }} The observational nature of the data raises questions about \textit{omitted variables}. \pause \vspace{0.15em} \\ \textcolor{gray}{\footnotesize $\Rightarrow$ Potential to experimentally manipulate sophistication} \medskip \vspace{0.25em}  \pause
    \item \alert{\textbf{Study design: }} The use of surveys allows \textit{direct measurement} of key variables but may simulate \textit{unrealistic} conditions. \pause \vspace{0.15em} \\ \textcolor{gray}{\footnotesize $\Rightarrow$ Replicate findings in more naturalistic setting, with additional outcome variables} 
\end{itemize} \medskip

\textbf{Questions} \vspace{0.25em}
\begin{itemize}
    \item \alert{\textbf{Model Specifications: }} any suggestions for alternative/additional \textit{model specifications}?  \pause 
    % \vspace{0.15em} \\ \textcolor{gray}%{\footnotesize $\Rightarrow$ Separate models for Democrats and Republicans, and for CRT and PK} 
    \medskip \vspace{0.25em}  \pause
    \item \alert{\textbf{Difference Between Concordant and Discordant Content: }} Is \textit{bias} in information processing the appropriate term? 
\end{itemize}


\end{frame}


\begin{frame}[plain]{}  
\begin{center}
    \huge{\textbf{Thank you!}} \bigskip

 \small{\textcolor{gray}{Gabrielle Péloquin-Skulski (gskulski@mit.edu) \\
 Chloe Wittenberg (cwitten@mit.edu) \\
 Adam Berinsky (berinsky@mit.edu) \\
 Gordon Pennycook (gordon.pennycook@cornell.edu) \\
 David Rand (drand@mit.edu)}}
\end{center}
\end{frame}

\appendix

\setlength{\leftmargini}{2pt}

\setbeamertemplate{section in toc}[default]
\setbeamertemplate{subsection in toc}[ball unnumbered]
\setbeamercolor{subsection in toc}{fg=gray}
\setbeamerfont{subsection in toc}{size=\scriptsize}
\setbeamerfont{section in toc}{size=\small}

\begin{frame}[t, fragile, label = app]{Supplementary Materials}
\vspace{1em}    
\tableofcontents
    
\end{frame}

\section{Data and Methods}


\begin{frame}[t,fragile, label = example_head]{Example Headlines}
\vspace{-0.25em}
\begin{figure}[!t]
\centering
\scalebox{0.75}{
\begin{tabular}{ccc}
\textit{(a) True, Democratic-Leaning} & \textit{(b) False, Democratic-Leaning}   \\ 
  \includegraphics[width=0.5\textwidth,valign=t]{figures_crt_pk/true_dem.jpg} &   
  \includegraphics[width=0.5\textwidth,valign=t]{figures_crt_pk/false_dem.jpg}\vspace{1em}
  \\

\textit{(c) True, Republican-Leaning} &\textit{(d) False, Republican-Leaning} \\
 \includegraphics[width=0.5\textwidth,valign=t]{figures_crt_pk/true_rep.jpg} &   \includegraphics[width=0.5\textwidth,valign=t]{figures_crt_pk/false_rep.jpg}\\[6pt]
\end{tabular}}
\label{fig:example_headlines} 
\end{figure}

\end{frame}


\subsection{Overview of Studies}

\begin{frame}[t, fragile, label = studies]{Overview of Studies}


\footnotesize{
\def\arraystretch{1.4}
\begin{table}[!t]
    \centering 
    \rowcolors{2}{gray!15}{white}
     \aboverulesep=0ex
 \belowrulesep=0ex
\begin{center}
\scalebox{0.625}{ 
\begin{tabular}{>{\centering \raggedright}m{0.02\textwidth}>{\centering}m{0.17\textwidth} >{\centering \raggedright}m{0.23\textwidth} >{\centering \raggedright}m{0.165\textwidth}  >{\centering \raggedright}m{0.46\textwidth}}
\toprule
\rowcolor{gray!15}
\textbf{ID} & \multicolumn{1}{c}{\centering \textbf{Timing}}  &  \multicolumn{1}{c}{\centering \textbf{Sample Size}} & \multicolumn{1}{c}{\centering \textbf{Vendor}}  & \multicolumn{1}{c}{\centering \textbf{Description}}   \tabularnewline \midrule
A & Feb. 2019 & $n$ = 399; 11,970 observations & Lucid & 30 headlines \textit{(10 true, 10 false, 10 misleading)} \tabularnewline  \midrule
B & June 2019 & $n$ = 3,880; 38,800 observations & Mechanical Turk/Lucid  & 210 headlines \textit{(70 true, 70 false, 70 misleading)}  \tabularnewline  \midrule
C & Aug. 2019 & $n$ = 493; 17,748 observations & Mechanical Turk & 36 headlines \textit{(18 true, 18 false)} \tabularnewline  \midrule
D & Aug. 2019 & $n$ = 284; 10,224 observations & Mechanical Turk & 36 headlines \textit{(18 true, 18 false)}  \tabularnewline  \midrule
E & Jan. 2020 & $n$ = 279; 10,036 observations & Mechanical Turk & 36 headlines \textit{(18 true, 18 false)}  \tabularnewline  \midrule
F & March 2020 & $n$ = 992; 17,856 observations & Mechanical Turk & 216 headlines \textit{(72 true, 72 false, 72 misleading)} \tabularnewline  \midrule
G & June 2022 & $n$ = 184; 2,198 observations & Bovitz, Inc. & 108 headlines \textit{(46 true, 58 false, 4 misleading)} \tabularnewline  \midrule
H & June/July 2022 & $n$ = 529; 6,312 observations & Lucid & 108 headlines \textit{(46 true, 58 false, 4 misleading)} \tabularnewline  \midrule
I & June/July 2022 & $n$ = 233; 2,796 observations & Mechanical Turk & 108 headlines \textit{(46 true, 58 false, 4 misleading)} \tabularnewline  \midrule
J & June 2022 & $n$ = 185; 2,220 observations & Prolific & 108 headlines \textit{(46 true, 58 false, 4 misleading)} \tabularnewline  \midrule
K & Sept. 2022 & $n$ = 203; 2,426 observations & Cloud Connect & 108 headlines \textit{(46 true, 58 false, 4 misleading)} \tabularnewline  \midrule
L & Sept. 2022 & $n$ = 218; 2,616 observations & Cloud Research & 108 headlines \textit{(46 true, 58 false, 4 misleading)} 
\tabularnewline \bottomrule 
\end{tabular}%
}
\end{center} 
\end{table}}

\vspace{-0.5em}
\hspace{-1em}\hyperlink{data_overview<4>}{\beamergotobutton{Main Slides}}\hspace{0.5em}\hyperlink{app}{\beamergotobutton{Appendix}}
    
\end{frame}


% \begin{frame}[t, fragile, label = summary_studies1]{Summary of the Studies}


% \vspace{-7em} 
% \begin{figure}
%     \centering
%     \includegraphics[width = 0.8\textwidth, angle=-90]{figures_crt_pk/summary_studies.pdf}
% \end{figure}

% \vspace{-2em}
% \hspace{-1.25em}\hyperlink{data_overview<6>}{\beamergotobutton{Main Slides}}\hspace{0.5em}\hyperlink{app}{\beamergotobutton{Appendix}}

% \end{frame}


\subsection{Cognitive Reflection Test Items}

\begin{frame}[t, fragile, label = crt_wording]{Cognitive Reflection Test (CRT) Items}

\setlength{\leftmargini}{15pt}
\setlength{\rightmargini}{20pt}
\vspace{0.5em} \small


 \small
        \begin{enumerate}
        \item The ages of Mark and Adam add up to 28 years total. Mark is 20 years older than Adam. How many years old is Adam? \textbf{- Answer: 4}
        \item If it takes 10 seconds for 10 printers to print out 10 pages of paper, how many seconds will it take 50 printers to print out 50 pages of paper? \textbf{- Answer: 10}
        \item On a loaf of bread, there is a patch of mold. Every day, the patch doubles in size. If it takes 40 days for the patch to cover the entire loaf of bread, how many days would it take for the patch to cover half of the loaf of bread? \textbf{- Answer: 39}
        \item If you’re running a race and you pass the person in second place, what place are you in? \textbf{- Answer: 2nd place}
        \item A farmer had 15 sheep and all but 8 died. How many are left? \textbf{- Answer: 8}
        \item Emily’s father has three daughters. The first two are named April and May. What is the third daughter’s name? \textbf{- Answer: Emily}
        \item How many cubic feet of dirt are there in a hole that is 3’ deep x 3’ wide x 3’ long? \textbf{- Answer: 0}

    \end{enumerate}
\vspace{1.5em}

\hspace{-0.75em}\hyperlink{variables<3>}{\beamergotobutton{Main Slides}}\hspace{0.5em}\hyperlink{app}{\beamergotobutton{Appendix}}

\end{frame}

\subsection{Political Knowledge Items}


\begin{frame}[t, fragile, label = pk_wording]{Sample of Political Knowledge Items}

\setlength{\leftmargini}{15pt}
\setlength{\rightmargini}{20pt}
\vspace{0.5em} \small


\small
    \begin{enumerate}
        \item ``Whose responsibility is it to decide if a law is constitutional or not?" \vspace{0.25em}
        \begin{itemize} \footnotesize
            \item Answer options: The President, Congress, The Supreme Court \vspace{0.5em}
        \end{itemize}
        \item ``Whose responsibility is it to nominate judges to federal courts?" \vspace{0.25em}
         \begin{itemize} \footnotesize
            \item Answer options: The President, Congress, The Supreme Court \vspace{0.5em}
        \end{itemize}
        \item ``Who is the Prime Minister of the United Kingdom? Is it:" \vspace{0.25em}
        \begin{itemize} \footnotesize
            \item Answer options: Theresa May, Angela Merkel, Tony Hayward, Richard Branson \vspace{0.5em}
        \end{itemize}
        \item ``Do you know what job or political office is currently held by Nancy Pelosi? Is it:" \vspace{0.25em}
        \begin{itemize} \footnotesize
            \item Speaker of the House, Treasury Secretary, Senate Majority Leader, Justice of the Supreme Court, Governor of New Mexico \vspace{0.5em}
        \end{itemize}
        \item ``Do you know what job or political office is currently held by Steve Mnuchin? Is it:" \vspace{0.25em}
                \begin{itemize} \footnotesize
            \item Attorney General, Justice of the Supreme Court, Treasury Secretary, House Republican Leader, Secretary of State \vspace{0.5em}
        \end{itemize}
    \end{enumerate}

\vspace{-0.25em}
\hspace{-0.75em}\hyperlink{variables<3>}{\beamergotobutton{Main Slides}}\hspace{0.5em}\hyperlink{app}{\beamergotobutton{Appendix}}

\end{frame}

\subsection{Summary Statistics}

\begin{frame}[t, fragile, label = summ_stats]{Distributions of Cognitive and Political Sophistication}


\vspace{-10em}
\begin{figure}
\hspace*{-4em}
    \centering
    \includegraphics[width = 0.95\textwidth, angle=-90]{figures_crt_pk/summary_statistics.pdf}
\end{figure}

\vspace{-3.75em}
\hspace{-0.75em}\hyperlink{data_overview<4>}{\beamergotobutton{Main Slides}}\hspace{0.5em}\hyperlink{app}{\beamergotobutton{Appendix}}
\end{frame}

% \subsection{Sample Demographics}


% \begin{frame}[t, fragile, label = sample_dem]{Sample Demographics}

% \begin{table}[ht]
% \centering
% \scalebox{0.6}{
% \begin{tabular}{llc}
%   \toprule
% {\textbf{Variable}} & {\textbf{Category}} & {\textbf{Percent}} \\ 
%   \midrule
% Gender & Male & 51.1\% \\ 
%    & Female & 48.9\% \\ 
%    \midrule
% Age & 18-24 & 10.4\% \\ 
%    & 25-34 & 32.2\% \\ 
%    & 35-44 & 23.0\% \\ 
%    & 45-54 & 14.2\% \\ 
%    & 55-64 & 11.1\% \\ 
%    & 65+ & 9.1\% \\ 
%    \midrule
% Education & No College Degree & 41.3\% \\ 
%    & College Degree & 58.7\% \\ 
%    \midrule
% Income & $<$\$50K & 51.2\% \\ 
%    & \$50K+ & 48.8\% \\ 
%    \midrule
% Party ID & Lean Democratic & 59.6\% \\ 
%    & Lean Republican & 40.4\% \\ 
%    \midrule
% Platform & Mechanical Turk & 54.1\% \\ 
%    & Lucid & 35.9\% \\ 
%    & Cloud Research & 5.3\% \\ 
%    & Prolific & 2.3\% \\ 
%    & Bovitz (Forthright) & 2.3\% \\ 
%    \bottomrule
% \end{tabular}
% }

% \label{tab:samp_perc}
% \end{table}

% \vspace{1em}
% \hspace{-1.25em}\hyperlink{variables<3>}{\beamergotobutton{Main Slides}}\hspace{0.5em}\hyperlink{app}{\beamergotobutton{Appendix}}

% \end{frame}



\section{Results}

\subsection{Descriptive Results}

\begin{frame}[t,fragile, label=descriptive]{Perceived Accuracy Varies by Veracity, Concordance}

\vspace{0.25em} 
\begin{figure}
    \centering
    \includegraphics[width = 0.9\textwidth]{figures_crt_pk/average_accuracy_real_concord.pdf}
\end{figure}

%\vspace{1.25em}

\end{frame}


\subsection{Disaggregating Average Ratings for False and Misleading Headlines}

\begin{frame}[t, fragile, label = disagg]{Disaggregating False and Misleading Headlines}

\vspace{0.25em} 
\begin{figure}
    \centering
    \includegraphics[width = 0.9\textwidth]{figures_crt_pk/acc_avgs_hyp.pdf}
\end{figure}

\vspace{-0.5em}
\hspace{-1em}\hyperlink{acc_veracity<2>}{\beamergotobutton{Main Slides}}\hspace{0.5em}\hyperlink{app}{\beamergotobutton{Appendix}}


\end{frame}

\subsection{Perceived Accuracy by Veracity}

\begin{frame}[t, fragile, label = perc_acc_veracity]{Perceived Accuracy Disagreggated by Veracity}

\begin{figure}
    \centering
    \includegraphics[width=0.9\textwidth]
    {figures_jdm/veracity_full_plot.pdf}
\end{figure}


\vspace{-1.8em}
\hspace{-0.75em}\hyperlink{acc_veracity<2>}{\beamergotobutton{Main Slides}}\hspace{0.5em}\hyperlink{app}{\beamergotobutton{Appendix}}

\end{frame}

\subsection{Disaggregating Truth Discernment}

\begin{frame}[t, fragile, label = dis_concord]{Truth Discernment Disaggregated by Concordance}

\begin{figure}
    \centering
    \includegraphics[width=0.9\textwidth]
    {figures_jdm/truth_discernment_conc.pdf}
\end{figure}


\vspace{-1.8em}
\hspace{-0.75em}\hyperlink{acc_veracity<2>}{\beamergotobutton{Main Slides}}\hspace{0.5em}\hyperlink{app}{\beamergotobutton{Appendix}}

\end{frame}

\subsection{Dissagregating Difference in Perceived Accuracy by Concordance}

\begin{frame}[t, fragile, label = decomp_veracity]{Bias Dissagregated by Veracity}

\begin{figure}
    \centering
    \includegraphics[width = 0.8\textwidth]{figures_jdm/conc_disc_veracity.pdf}
\end{figure}

\vspace{-1.2em}
\hspace{-0.75em}\hyperlink{acc_conc<2>}{\beamergotobutton{Main Slides}}\hspace{0.5em}\hyperlink{app}{\beamergotobutton{Appendix}}

\end{frame}

\end{document}